\chapter{Lecture 14}
\lhead{February 13, 2015}
\chead{21-366 Lambda Calculus Lecture 14}

\section{Valuation on a Term $X$}
\index{Valuation|textbf}
During the previous lecture we mentioned a function which mapped terms to nonnegative integers. Now we formalize that idea as the valuation of a term, and give some restrictions on what forms a valuation can take. A valuation on a term is a map $\v : \hbox{subterms}(X) \rightarrow \mathbb{N}^+$ satisfying two conditions:

\begin{itemize}
\item The value $\v(x) > \sum_{T\hbox{ to the right of }x}\v(T)$ in a redex in $R$. So for $(\l uU)V$, $\v(u) > \v(U)$.
\item $\v((UV)) > \hbox{max}(\v(U),\v(V))$ and $\v(\l uU) > \v(U)$
\end{itemize}

Valuations exist for any term. Assume by recursion that
\begin{itemize}
  \item All leaves to the right of $x$ have values
  \item All subterms all of value variable occurance have values themselves have values. (What?)
\end{itemize}

Given a valuation of $X$,
\begin{equation*}
  \# X = \sum_{Y\hbox{ a subterm of }X}\v(Y)
\end{equation*}

\subsection{Colored Reductions}
\index{Colored Reduction|textbf}
Select some subset $R$ of the set of redexes in a given term $X$ and color them green. Color all others red. We defined last time colored reductions $X \rightarrow X_o {{\Delta_1} \atop {\rightarrow_\beta}} X_1 {{\Delta_2} \atop {\rightarrow_\beta}} \ldots$ where $\Delta_1 \in R$, and $\Delta_2 \in$ residuals of $R$.\\

\textbf{Claim:} There exists a valuation on $X_1$ induced by the reduction $X_0 {{\Delta_1} \atop {\rightarrow_\beta}}$ such that $\# X_1 < \# X_0$.\\

\textbf{Proof:} Consider a redex $\Delta_1 \in R$...\\

\textbf{Conclusion:} Every colored reduction sequence terminates. This gives us that every colored reduction sequence terminates in a unique ``normal form'' with no colored redexes. We call these colored reductions ``complete development.''

\subsection{The Lemma of Parallel Moves}
\index{Strip Lemma|textbf}
This is also known as the Strip Lemma. If we have a reduction of the form
\begin{center}
  \begin{tikzpicture}
    \draw (-1,-1) -- (0,0) -- (1,-1); % Left is many reductions, Right is one.
    \draw[dashed] (-1,-1) -- (0,-2) -- (1,-1);
  \end{tikzpicture}
\end{center}

\textbf{Remark:} Church-Rosser follows by induction on the conversion. With many and one, we can repeatedly apply until we get many and many.\\

\subsection{Proof of the Strip Lemma}
