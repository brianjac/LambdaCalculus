\chapter{Lecture 11}
\lhead{February 6, 2015}
\chead{21-366 Lambda Calculus Lecture 11}
\rhead{Brian Jacobs}
\pagestyle{fancy}

\section{Even More Data Structures}

\subsection{Stacks}
The empty stack is defined as $KK_*$, or THREE. If our stack is $\l a\ aLR$, $L$ is the top, $R$ is the result of popping the stack.

\subsection{Trees}
What will we do today?
\begin{enumerate}[(I)]
  \item Describe the data structure
  \item Tasks:
    \begin{itemize}
      \item insertion of an element
      \item balancing the tree
      \item finding an element in the tree
    \end{itemize}
  \item write an informal algorithm for finding an element in a tree
    \begin{itemize}
      \item write a lambda term which implements our algorithm
    \end{itemize}
\end{enumerate}

A tree is a data structure which is comprised of leaves and internal nodes, where a leaf is defined as an ordered triple $<\hbox{TRUE}, \uline{n}, X>$ of a boolean which is true, an index number, and an object. An internal node is an ordered triple of $<\hbox{FALSE}, L, R>$ of a false boolean, a left subtree, and a right subtree.\\

We provide a recursive definition of trees: $<\hbox{TRUE}, \uline{n+1}, X>$ is a tree, and $<\hbox{FALSE}, L, R>$ is a tree.\\

Now we describe our algorithm: $f(x,y,z)$ where $x$ is a tree, $z$ is the item we are searching for, and $y$ is a stack of trees. If $x$ is a leaf, then $x(TWO)$ is the index. We apply $Eq(x(TWO))z$. If this is $\uline{1}$, then we want to return $x$ of 3, so we write:
\begin{equation*}
  Y_{Turing}(x(ONE)(Eq(x(TWO))z(x(THREE))(f\ (\hbox{TOP }y)(\hbox{POP }y)z)(f (x(TWO))(\l a\ a(x(THREE))y))))
\end{equation*}
We call this function as $Fx(\hbox{Empty stack}z)$.\\