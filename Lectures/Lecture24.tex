\chapter{Lecture 24}
\lhead{March 18, 2015}
\chead{21-366 Lambda Calculus Lecture 24}
\rhead{Brian Jacobs}

\section{Diamond Property for Reductions}
\subsection{Beta Reduction}
\subsection{Eta Expansion}
\begin{eqnarray*}
  U^{\alpha,n+1,n+2,\beta} &\rightarrow& (\l u^{n+1,n}(U^{\alpha,n+1}u^{n+1,n})^{n,n+1})^{n+2}\\
  U^{\alpha,n+2,n+1,\beta} &\rightarrow& (\l u^{n,n+1}(U^{\alpha,n+2}u^{n,n+1})^{n+1,n})^{n+1}
\end{eqnarray*}

\subsection{Potential Reductions}
\begin{eqnarray*}
  (U^{\alpha,n+1,n+2}V^{\beta,n+1})^{n+1,\gamma} &\rightarrow& (U^{\alpha,n+1}V^{\beta,n+1,n})^{n,n+1,\gamma}\\
  (U^{\alpha,n+2,n+1}V^{\beta,n})^{n,\gamma} &\rightarrow& (U^{\alpha,n+2}V^{\beta,n,n+1})^{n+1,n,\gamma}\\
  (\l u^{n,}U^{\alpha,n})^{n+1,n+2,\gamma} &\rightarrow& (\l u^{n+1,n}U^{\alpha,n,n+1})^{n+2,\gamma}\\
  (\l u^{n+1}U^{\alpha,n+1})^{n+2,n+1,\gamma} &\rightarrow& (\l u^{n,n+1}U^{\alpha,n+1,n})^{n+1}
\end{eqnarray*}
To prove the diamond property for reductions, any reduction tree without potential can be given potential.
\begin{enumerate}[(1)]
  \item Every $\beta$ reduction sequence terminates. (add clockwise rotation)
  \item Every eta expansion sequence terminates.
  \item Every potential reduction sequence terminates.
\end{enumerate}

\section{Tools for Proving Termination}
\subsection{Multiset Ordering}
A multiset is a set with multiplicity. Vaguely, a multiset is a finite set of natural numbers where each integer can occur more than once. The simplest example would be the set $\{n,n\}$, where the integer $n \in \mathbb{Z}$. A multiset is an assignment of nonnegative integers to the nonnegative integers, i.e. a function $f:\mathbb{N}\mapsto\mathbb{N}$ such that there exists $m \in \mathbb{N}$ such that for all $n \geq m, f(m) = 0$. A third definition: A multiset is a finite list of nonnegative integers modulo order.\\

We can order multisets as follows: $X > Y$ if there is an element $x \in X$ and a multiset $Z$ such that for each $z \in Z$, $z < x$ and $Y = X \setminus \{x\}$. That is to say, we say $X = \{\ldots,x,\ldots\}$ and $Z = \{z_1,\ldots,z_k\}$ for some $k \in \mathbb{N}$ and then do $X = \{\ldots,z_1,\ldots,z_k\ldots\}$.\\

\textbf{Theorem:} There is no infinite sequence of multisets $X_0 > X_1 > X_2 > \ldots > X_k > \ldots$. Recall K\"onig's Lemma: A tree which is finitely branching and infinite has an infinite path.\\

Suppose that we have an infinite descending sequence $X_n$ of multisets. A finitely branching tree of nonnegative integers. At any given stage the current $X_i = $ the leaves of this tree. Note that this tree has infinite nodes. However by K\"onig's lemma, this finitely branching tree must have an infinite path. This is a contradiction.\\