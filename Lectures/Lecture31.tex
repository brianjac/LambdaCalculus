\chapter{Lecture 31}
\lhead{April 3, 2015}
\chead{21-366 Lambda Calculus Lecture 31}
\rhead{Brian Jacobs}
\pagestyle{fancy}

\section{Proof of B\"ohm's Theorem}
\textbf{Recall:} $M$,$N$ in $\beta$-normal form. $M \not=_\beta N$.\\

\textbf{Claim:} There exists $P_1\ldots P_\ell$ such that
\begin{eqnarray*}
  MP_1\ldots P_\ell K\\
  NP_1\ldots P_\ell K_*\\
\end{eqnarray*}
\textbf{Remark:} By $\eta$ expansions we can assume that the B\"ohm trees of $M$ and $N$ are the same ``abstract'' tree. That is, the trees are the same without labels.\\

\textbf{Proof:} The proof is by induction on the depth of the first place where these two B\"ohm trees actually differ.\\

The basis case is when the two B\"ohm trees differ at the root, i.e. $M = \l x_1\ldots x_m.x_iX_1\ldots X_t$ and $\l y_1 \ldots Y_n . y_j Y_1\ldots Y_t$.\\

There are two cases.
\begin{enumerate}[(1)]
  \item $n \not= m$ so we assume without loss of generality that $m > n$
\end{enumerate}
We define
\begin{equation*}
  K_*^k := \l z_1\ldots z_k I
\end{equation*}
so that $K_*^1 = K_*$.
We then write for $r \geq t$
\begin{equation*}
  M\underbrace{K_*^{r}\ldots K_*^{r}}_{m} =_\beta K_*^{r}X_1^+\ldots X_t^+ =_\beta K_*^{r-t}
\end{equation*}
And
\begin{equation*}
  N\underbrace{K_*^{r}\ldots K_*^{r}}_{m} =_\beta K_*^{r}Y_1^+ \ldots Y_t^+ K_*^{r} \ldots K_*^{r} \rightarrow_\beta K_*^{r-t}\underbrace{K^r\ldots K^r}_{m - n} =_\beta K_*^{r-t+m-n}
\end{equation*}
Pick $r$ such that these two terms reduce to terms of the form $K_*^\alpha$ and $K_*^\beta$ where $\alpha \not= \beta$.\\

Assume $\alpha > \beta$. It suffices to find $Q_1\ldots Q_s$ such that $K_*^{\alpha}Q_1\ldots Q_s = K$ and $K_*^{\beta}Q_1\ldots Q_s = K_*$. Explicitly, these are $\l x_1\ldots x_\alpha . I Q_1\ldots Q_s$.\\

\textbf{Induction step:} Let
\begin{eqnarray*}
  M = \l x_1\ldots x_n . x_i X_1\ldots X_t\\
  N = \l x_1\ldots x_m . x_i Y_1\ldots Y_t\\
\end{eqnarray*}
Let $r$ be the smallest $1 \leq r \leq t$ such that the first place at which they differ is in the pair $X_rY_r$. Consider $x_iX_1\ldots X_t$ and $x_iY_1\ldots Y_t$. Pick an integer $\ell$ which is very large, i.e. $\ell >> t + t$. Substitute $\l y_1\ldots y_\ell \l a. a y_1\ldots y_\ell$ for $x_1$.