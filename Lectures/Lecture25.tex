\chapter{Lecture 25}
\lhead{March 20, 2015}
\chead{21-366 Lambda Calculus Lecture 25}
\rhead{Brian Jacobs}

\section{Localization}
There are two types of local rotations, one for beta and one for eta.

\begin{enumerate}[(1)]
  \item (Eta) Given an eta expansion $U \rightarrow_\eta \l u(Uu)$ followed by any sequence of rotations, then the rotaions can be done first on the term containing just $U$. Except for local clockwise rotations. Those are of the following form, for example:
\end{enumerate}
\begin{eqnarray*}
  (\l u^{n,} (U^{\alpha,m+1,m+2}u^{n,\beta,m+1})^\square)^\diamond\\
  ((U^{\alpha,m+1}u^{n,\beta,m+1,m})^{m,\square})^\diamond
\end{eqnarray*}
Local means a clockwise rotation of the particular application $U \rightarrow_\eta \l u(Uu)$.\\

(2) (beta) Given beta reduction $(\l uU)V \rightarrow_\beta [V/u]U$. preceded by a seqnece of rotations. Then all the rotations can be done last, except local clockwise rotations of the redex.
\begin{eqnarray*}
  (\l u^{n,}U^{\alpha,n})^{n+1,n+2}V^{\delta,n+1}\\
  \rightarrow (\l u^{n+1,n}U^{\alpha,n,n+1})^{n+1}V^{\delta,n+1}\\ % Result of a counterclockwise rotation
  = [V^{\delta,m+1}/u^{n+1,n}]U^{\alpha,n,n+1}\\ % Further reduction
  ...\\
  \rightarrow (\l u^{n,}U^{\alpha,n})^{n+1}V^{\delta,n+1,n}\\ % Result of a clockwise rotation
  = [V^{\delta,n+1,n}/u^{n,}]U^{\alpha,n}
\end{eqnarray*}

\subsection{Eta Postponement}
If we have an eta expansion followed by a local clockwise rotation followed by a beta reduction, then these can be done in the ``opposite'' order: a local clockwise reduction followed by a beta reduction followed by a sequence of eta expansions.
\begin{equation*}
  (\l uU)V \rightarrow [V/u]U
\end{equation*}
Say that we have an eta expansion inside a term, $U \rightarrow \l uUu$, followed by some number of rotations local to the eta, followed by a beta reduction $(\l yY)z \rightarrow [Z/y]Y$.

% \Delta_1 : U -> \l u Uu -->> down(marked local to the eta) to \Delta_2 (\l yY)z -> [Z/y]Y

\textbf{Case 1:} $\Delta_1,\Delta_2$ are disjoint.\\
\textbf{Case 2:} Overlap:
2(a) $\Delta_1 \subseteq \Delta_2$ if $\Delta_1 = \Delta_2$ then we are clearly done in opposite order. If $\Delta_1 \subset \Delta_2$ and $\Delta_1 \subseteq Y$ $\Delta_1 \subseteq Z$ maybe many eta expansions in $[Z/y]Y$. $\Delta_1 = \l yY$ i.e. $UZ, \l u(Uu)Zm UZ$. Then simply keep $UZ$.\\

2(b) $\Delta_2 \subseteq \Delta_1$. There are several subcases: either $\Delta_2 = Uu$ or $\Delta_2 \subseteq U$ (This is another easy case). Now if $\Delta_2 = Uu$, then we get
\begin{eqnarray*}
  \l yY\\
  \l u((\l yY)u)\hbox{ clockwise rotations of $\l yY$ potentials onto $u$.}\\
  \l u([u/y]Y)\hbox{ net effect is a counterclockwise rotation}
\end{eqnarray*}

\textbf{Proof of Termination:} % TODO: IMPORT PROOF OF TERMINATION FROM WEBSITE
We need to show that:\\

(1) Every alternating sequence of beta reduction and clockwise rotations terminates.\\

(2) Every sequence of rotations terminates.\\

(3) Every sequence of eta expansions and clockwise rotations terminates.
\begin{eqnarray*}
  U^{\alpha,n+2,n+1,\beta}\rightarrow(\l u^{n,}(U^{\alpha,n+2}u^{n,n+1})^{n+1,n})^{n+1,\beta}
\end{eqnarray*}
We assign for each subterm the sum of the values of the potential. The value of the term equals the multiset of those sums.\\

Suppose we have an infinite sequence of betas, etas, and rotations. Take the first beta, and permute it to the beginning. You now have a number of local rotations, followed by a beta. We repeat this process until we have a sequence of local rotations, followed by a sequence of betas, followed by etas and rotations. This terminates, obviously. For the full, formal proof, see the appendix.