\chapter{Lecture 35}
\lhead{April 13, 2015}
\chead{21-366 Lambda Calculus Lecture 35}
\rhead{Brian Jacobs}
\pagestyle{fancy}

\section{Replacement and Substitutions}
\subsection{Kleene's Evaluator}
\begin{equation*}
  E_{Kleene}\godnum{M} =_{\beta} M
\end{equation*}
for some closed term $M$. $E$ is defined by recursion with a parameter $X$ which is a stack of the form $\langle \uline{n}, Z\rangle$ which tells us to substitute the term $Z$ for the free variable $x_n$. In pidgeon lambda calculus:\\

$F\godnum{Y}X =$ if $P = K_1^3$, then find $S$ in $X = Z$ and return it if it exists else return some random term, say $\Omega$. If $P = K_2^3$, then recurse with $(FRX)(FSX)$. If $P_3^3$, then $F\godnum{Y}X$ = $\l u.FS$Push$\langle R,u\rangle$.\\

Consider that things lower in the recursion will be accessed in the stack first. So the same variable can be bound int two different spots without problems.\\

We then define $F$ as a fixed point and set $E = \l a.Fa\bot$. For true codes, we can see that $E\godnum{M} =_\beta M$.\\

\subsection{Replacement}
We assume that $\godnum{X}$ has comments corresponding to depth first search. We want to define a term such that $REP\godnum{X}\godnum{Y}\uline{n}$ corresponds to the replacing of the subterm $X$ with dfs number $n$ by $Y$.\\

So if $P = K_1^3$, then if $Q = n$, return $\godnum{Y}$, and if $Q \not= n$, return $\godnum{X}$.

\subsection{Substitution}
$SUB\godnum{X}\godnum{Y}\uline{n} = 0$ substiutute $Y$ in $X$ for all free cocurances of $x_n$ in $X$. A question arises when bound variables are considered. $Y$ should be free for $x_n$ in $X$. $CV\godnum{X}\uline{n}$ is\footnote{$CV$ stands for conflicting variables.} a stack of variables bound on the paths to free occurances of $x_n$. If $X$ is a variable $x_k$, if $k = n$, then $\godnum{Y}$ else $\godnum{x_k}$.
