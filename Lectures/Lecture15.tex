\chapter{Lecture 15}
\lhead{February 16, 2015}
\chead{21-366 Lambda Calculus Lecture 15}

\section{Review}
The finiteness of developments results in a unique ``normal form.''\\

\textbf{Strip Lemma (the Lemma of Parallel Moves):}\index{Strip Lemma} A single reduction $\Delta$ and a $\beta$ reduction followed by a complete reduction of all the residuals of $\Delta$ can be reached by $\beta$ reducing the term after applying $\Delta$.
% Picture line 1: <<---- {\Delta \atop --->}
% Picture line 2: {complete reudction of all residuals of \Delta \atop ---->>} \dashed <<----

\section{Cofinal Reduction Strategies}
Assign to each term a reduction sequence
\begin{equation*}
  x = x_1 \rightarrow_\beta x_2 \rightarrow_\beta x_3 \rightarrow_\beta \ldots
\end{equation*}
So any $Y$ such that $X =_\beta Y$ is $\beta$ reduceable to some term in the sequence.\\

\subsection{Klop's Theorem}
\index{Klop's Theorem}
\textbf{Theorem (Jan Willem Klop):} A reduction sequence $x = x_0 \rightarrow_\beta x_1 \rightarrow_\beta \ldots$ is cofinal if for any $\Delta$ redex in any $x_i$, eventually there is a $j > i$ such that $\Delta$ has no residual in $x_j$.\margnot{Jan Willem Klop}{ Klop was born in 1945 in Gorinchem, Netherlands. He works on the Algebra of Communicating Processes at Vrije Universiteit, in Amsterdam.}\\

\textbf{Correlary:} The following sequence is cofinal:
\begin{equation*}
  x_1 \twoheadrightarrow_{\beta\hbox{ complete reduction of all redexes in $x_1$}} x_2 \twoheadrightarrow_{\beta\hbox{ complete reduction of all redexes in $x_2$}} \ldots
\end{equation*}

\textbf{Proof:} Suppose that the reduction sequence
\begin{equation*}
  x = x_0 \rightarrow_\beta x_1 \rightarrow_\beta \ldots
\end{equation*}
has the quality that for a redex $\Delta$ in $x_i$ there exists some $x_j$, $j > i$ such that $\Delta$ has no residual in $X_j$. Therefore, there exists some $k > i$ such that there are no redexes in $x_i$ have residuals in $x_k$. Now we prove that this reduction sequence is cofinite. Given $Y =_\beta x =_\beta x_0$, then by Church-Rosser, there exists a $Z$ such that $X \twoheadrightarrow_\beta Z \twoheadleftarrow_\beta Y$. We will show that there exists an $i$ such that $Z \twoheadrightarrow_\beta X_i$. We have that
\begin{equation*}
  Z \stackrel{\mathscr{R}}{\twoheadleftarrow_\beta} x_0 \rightarrow x_1 \rightarrow \ldots \rightarrow x_i \rightarrow \ldots
\end{equation*}

We induct on the length of the reduction sequence $\mathscr{R}$ from $X$ to $Z$. Basis: take the one redex $\Delta$ in $\mathscr{R}$. Let $i = 0$. Then there exists $j > 0$ such that $X_j$ has no residuals of $\Delta$ by the strip lemma.\\

We now consider for our induction step that $\mathscr{R}$ has a length greater than $1$.
\begin{equation*}
  \ldots \leftarrow_\beta x_i \leftarrow_\beta \ldots \leftarrow_\beta x_1 \leftarrow_\beta x_0 \twoheadrightarrow_\beta U \rightarrow_\beta(\Delta)\ Z
\end{equation*}
Take the last reduction, $\Delta$ in $\mathscr{R}$. By the induction hypothesis, there exists an $n$ such that $U \twoheadrightarrow X_n$. Let $C$ be the set of residuals of $\Delta$ in $X_n$. There exists a $k > n$ such that no ersidual of $C$ has $X_k$ a residual in $X_n$.\\

\section{A Method to Prove that $X \not=_\beta Y$}
We will generalize colored reductions. The idea, due to J.~J.~Levy, is called Levy labels.\index{Levy labels}\\

The general idea of Levy labels is that if we have a term $X$, we assign nonnegative integers to the subterms of $X$, representing the ``potential to reduce.'' Colored reduction simply assigned all ``red'' redexes $0$, and ``green'' redexes $1$, saying they had the potential for one reduction.\\

For our purposes, we will instead use finite sequences of integers. These sequences are nonempty, and the integers themselves are nonnegative. These sequences additionally obey the following property: $n_{i + i} = n_i \pm 1$.

\begin{equation*}
  n_1,n_2,n_3,\ldots n_k
\end{equation*}
\begin{itemize}
  \item $n_i \geq 0$
  \item $n_{i + i} = n_i \pm 1$
  \item $k > 0$
\end{itemize}
If $Y$ is a subterm of $X$, we will write $Y = Y^\alpha$, where $\alpha = n_1\ldots n_k$. We require that all occurances of a given variable $X$ should always have the same first element in the sequence $x = x^{n\alpha}$. The $\alpha$s may vary. Applications look like:
\begin{equation*}
  (U^{\alpha,{n+1}},V^{\beta,n})^{n,\gamma}
\end{equation*}
Abstractions look like:
\begin{equation*}
  (\l x^{n} X^{\alpha,n})^{n + 1,\gamma}
\end{equation*}